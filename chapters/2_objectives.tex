\chapter{Objectives}
\label{ch:objectives}
These objectives will be cleanly divided into core work required to produce a complete dissertation within a 6-month timeframe, followed by a rough outline of follow-up work time-permitting.

\section{Must-have Objectives}

\subsection{Foundational Literature Review and Protocol Selection}
Initially, I will conduct a thorough, focussed literature review (\cite{https://doi.org/10.4230/lipics.itcs.2024.24}, \cite{caro2025interactiveproofsverifyingquantum}, \cite{mahadev2023classicalverificationquantumcomputations} and related and cited works) to map the landscape of interactive proofs for quantum learning. This will identify and summarize foundational protocols, and more importantly, analyze recent, resource-efficient proposals designed for near-term application.

A strong candidate (\cite{ma2024classicalverificationquantumlearning}, for example) would be a protocol with a classical verifier or one that explicitly addresses noise - this should approximatly occupy the first month of work.

\subsection{Theoretical Analysis of the Selected Protocol}
Once a protocol is selected, I will perform a detailed theoretical investigation and analysis of its components and resource requirements under ideal conditions, breaking down the protocol into its constituent steps and quantifying computational loads. This will include runtime complexity, (quantum) memory overhead, and the magnitude of data communicated over its channel.

\subsection{Simulation and Performance Analysis under a Realistic Noise Model}
This is the main practical component of the project, implementing a simplified, small-scale version of a previously unimplemented protocol (for example the recent work from Caro et. al. \cite{https://doi.org/10.4230/lipics.itcs.2024.24}). I will then introduce a standard noise model to the simulation, the goal being to run experiments that measure how the protocol's key properties (completeness and soundness) degrade as noise increases. This should also occupy a month of time.

One method of noise generation I could explore is the depolarizing channel \cite{king2002capacityquantumdepolarizingchannel}, also implemented by \texttt{PennyLane} \cite{pennylane-dp}.

\subsection{Proposing and Simulating a Protocol Modification}
Based on insights from simulation and theoretical analysis, I will propose a series of modifications to the protocol aimed at improving its noise tolerance or resource efficiency. I will then both prove its validity theoretically when appropriate and through simulation too. I estimate this will take around 3 months.

\section{Could-have Objective}

\subsection{Theoretical Framework for an Adaptive Protocol}
Finally, as a path of extension, I will aim to develop an adaptive protocol which can alter its parameters or behaviour based on network conditions. This could be from a purely theoretical or potentially more practical standpoint. This will take around 3 months, generously, but I will be able to write about it before completion from a development standpoint if uncompleted.
