\chapter{Project Management \& Reflection}
As a research project, this did not have a "formal methodology" as typical software engineering products do. Instead, I opted for biweekly communication where possible with my supervisor Dr. Matthias Caro, and adherance as best possible to my original specification timetable.
\section{Alignment with original timelines}
Under normal circumstances, I would indeed be slightly behind schedule for this project, given that it is worth 300 hours of work I have roughly completed $\sim 110$. However, due to unfortunate circumstances, I was ill for a significant portion of winter and term-time, meaning I was unproductive (and bed-bound) for several weeks. Whilst this was extremely detrimental for progress, I made attempts to recover time by allocating additional hours at the start of the second term in order to catch up, plus additional meetings to get me up to speed. 
\subsection{Updated timetable}  
Despite setbacks, I was able to stick basically on schedule (including originally having the date wrong for submission of the progress report). The next few weeks of work are also clearly outlined, covering both a theoretical and practical path of progression mentioned earlier in this report.

It should also be noted that last term, I did not keep a consistent schedule of work for this module, and especially working under the assumption I could complete work over winter, I was too relaxed about timings. Since the start of this term, I have been putting in a mean of 3 hours a day into this project, and will continue to do so until completion. This puts me on track for the target hours of the module.
\begin{table}[H]
	\centering
	\caption{Updated Project Timetable}
	\label{tab:timetable}
	\begin{tabular}{@{}l|l@{}}
		\toprule
		\textbf{Week}      & \textbf{Event / Task}                                             \\ \midrule
		Summer – T1 W1     & Initiation of project. Background reading on quantum computing    \\
		                   & and interactive proof systems with use cases.                     \\ \addlinespace
		T1 W2              & Submission of project specification document.                     \\ \addlinespace
		T1 W3–W6           & Complete a thorough, focussed literature review to map the        \\
		                   & landscape of interactive proofs for quantum learning, including   \\
		                   & in noisy environments.                                            \\ \addlinespace
		T1 W7–W8           & Investigate and select the strongest candidate protocol with a    \\
		                   & classical verifier for both theoretical and practical analysis.   \\ \addlinespace
		T1 W9              & Sidetracked and explored in more detail the mathematics behind    \\ 
                       & TCF functions and its use in classical verification               \\ \addlinespace
    T1 W10             & Ended up ill into the holidays                                    \\ \midrule
		Christmas Holidays & Caught up on reading for other subjects, fell ill again at the    \\ 
                       & end of the holidays!                                              \\ \midrule
		T2 W1              & Begin theorizing potential protocol modifications.                \\ \addlinespace
		T2 W2              & Write and submit progres report                                   \\ \addlinespace
		T2 W3–W6           & Theoretically and practically analyze a series of candidate       \\
		                   & modifications.                                                    \\ \addlinespace
		T2 W6-W8           & Iterate and improve on previous results. Collect data.            \\ \addlinespace
		T2 W9              & Submission of mid-term progress presentation.                     \\ \addlinespace
		T2 W10             & Verify all results.                                               \\ \midrule
		Easter Holidays    & Optional exploration of new protocols. Complete final report.     \\ \midrule
		T3 W1              & Submission of final report.                                       \\ \bottomrule
	\end{tabular}
\end{table}

\section{Challenges encountered and resolutions}
Having already spoken about project management challenges, it is worth noting that likely my biggest setback and something acknowledged in risk analysis was the lack of prerequisite knowledge to enter the field. I should have allocated more time to learning the fundamentals early on, and exploring the literature in more depth to have a firmer grasp on next steps earlier on. 

Since the start of this second term, one of the good consequences of spending more time with the project is that I can dedicate hours to learning about the field. This makes all of my work better, from the practical implementations, to known best practices, to proof structure and quality of suggested improvements to protocols. 
